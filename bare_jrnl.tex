\documentclass[14pt, conference]{IEEEtran}
\ifCLASSINFOpdf
\else
\fi

\IEEEoverridecommandlockouts

\usepackage{listings}
%\usepackage{xcolor}
\usepackage[table, svgnames]{xcolor}
\definecolor{codegreen}{rgb}{0,0.6,0}
\definecolor{codegray}{rgb}{0.5,0.5,0.5}
\definecolor{codepurple}{rgb}{0.58,0,0.82}
\definecolor{backcolour}{rgb}{0.5,1,0.5}
 
\lstdefinestyle{mystyle}{
    backgroundcolor=\color{backcolour},   
    commentstyle=\color{codegreen},
    keywordstyle=\color{magenta},
    numberstyle=\tiny\color{codegray},
    stringstyle=\color{codepurple},
    basicstyle=\footnotesize,
    breakatwhitespace=false,         
    breaklines=true,                 
    captionpos=b,                    
    keepspaces=true,                 
    numbers=none,                    
    numbersep=5pt,                  
    showspaces=false,                
    showstringspaces=false,
    showtabs=false,                  
    tabsize=2
}
 
\lstset{style=mystyle}


\usepackage{float}
\usepackage{longtable}
\usepackage{graphicx}
\usepackage{multirow}
%\usepackage{subcaption}
\usepackage{flushend}
%\usepackage{hyperref}
\usepackage{tabularx}
\usepackage{booktabs} % For formal tables
\usepackage{hhline}
\usepackage{array}

\colorlet{headercolour}{DarkSeaGreen}
\AtBeginEnvironment{tabular}{\rowcolors{1}{\ifnumequal{\rownum}{1}{headercolour}{white}}{}}%

\newcolumntype{L}[1]{>{\raggedright\let\newline\\\arraybackslash\hspace{0pt}}m{#1}}
\newcolumntype{C}[1]{>{\centering\let\newline\\\arraybackslash\hspace{0pt}}m{#1}}
\newcolumntype{R}[1]{>{\raggedleft\let\newline\\\arraybackslash\hspace{0pt}}m{#1}}
\bibliographystyle{IEEEtran}

\usepackage{cite}
\usepackage{amsmath,amssymb,amsfonts}
\usepackage{algorithmic}
\usepackage{graphicx}
\usepackage{textcomp}
\def\BibTeX{{\rm B\kern-.05em{\sc i\kern-.025em b}\kern-.08em
    T\kern-.1667em\lower.7ex\hbox{E}\kern-.125emX}}

%\hypersetup{bookmarks=false}

\begin{document}

\title{Network Anomaly Detection Using LightGBM: \\ A Gradient Boosting Classifier}

\author{
\IEEEauthorblockN{Md. Khairul Islam\textsuperscript{1},
Prithula Hridi\textsuperscript{1}, Md. Shohrab Hossain\textsuperscript{1}, Husnu S. Narman\textsuperscript{2}}

\IEEEauthorblockA{\textsuperscript{1}Department of Computer Science and Engineering, Bangladesh University of Engineering and Technology, Bangladesh\\
    \textsuperscript{2}Weisberg Division of Computer Science, Marshall University, Huntington, WV, USA\\}
\IEEEauthorblockA{Email:  khairulislamtanim@gmail.com, prithula5117@gmail.com, mshohrabhossain@cse.buet.ac.bd,  narman@marshall.edu}
}

\maketitle

\begin{abstract}
Anomaly detection system plays a significant role in recognizing intruders or suspicious activities inside the system, detecting unseen and unknown attacks. In this paper, we have worked on a benchmark dataset UNSW-NB15, that
reflects modern-day network traffic. Previous works on this dataset lacked proper validation approach, followed only one evaluation setup which made it difficult to compare their results with others using same dataset with different validation steps. In this work, we have used a machine learning classifier LightGBM to perform binary classification on the dataset. We have performed different experimentation setups. Using ten-fold cross-validation on the train,
test and combined (training and test) dataset, our models have achieved 97.21\%, 98.33\% and 96.21\% f1\_scores, respectively. Also, the
 model fitted only on train data, achieved 92.96\% f1\_score on the separate test set. We have presented complete comparisons with previous works using all performance metrics. We have shown our model outperformed them in most metrics and thus can be used to detect network anomalies better.
\end{abstract}

\begin{IEEEkeywords}
anomaly detection, machine learning,  network security.
\end{IEEEkeywords}
%------------------------ Into \input{introduction.tex}

\section{Introduction}
As web applications are getting increasingly popular, the Internet has become a necessity in our day-to-day life. As a
consequence, network systems are being targeted more by attackers with malicious intents. To detect intruders in a
network system, there are generally two approaches: signature-based and anomaly-based detection.  maintain a database of previously known attacks and raise alarms when any match is found with the analyzed
 data. However, Signature-based
detection techniques  are vulnerable to zero-day attacks.

An anomaly in a network means a deviation of traffic data from its normal pattern. Thus, anomaly detection techniques
have the advantage of detecting zero-day attacks. However, in a complex and large network system,  it is not easy to define a set of valid requests or normal behavior of the endpoints. Hence, anomaly detection faces the disadvantage of
having a high false-positive error rate (events erroneously classified as attacks). There are different types of anomalies that can be mapped with different types of attacks. According to Ahmed et al.\cite{ahmed2016survey},
the main attack types are DoS, Probe, User to Root (U2R) and Remote to User (R2U) attacks. They\cite{ahmed2016survey} mapped the point
anomaly with the U2R and the R2U attacks, DoS attack to the collective anomaly and Probe attack to the contextual anomaly.

In recent years, machine learning and deep learning have become increasingly popular. They have been applied to
different anomaly and intrusion detection systems %\cite{fernandes2019comprehensive}, \cite{chalapathy2019deep}.
In many cases, they have outperformed the previous state-of-the-art models.

As UNSW-NB15\cite{moustafa2015unsw} is a benchmark anomaly detection dataset, numerous studies have been performed on it.
However, as shown in Section \ref{relatedWorks}, to evaluate the same dataset different setups were adopted.
Also, works that followed the same experimentation setup did not compare their results with prior works in some cases (for example, \cite{Kanimozhi2019UNSW-NB15} and  \cite{nawir2019effective} did not compare their results with \cite{koroniotis2017towards}), thereby making it difficult to validate their improvements. There are several works~
%\cite{mogal2017nids}
\cite{Kanimozhi2019UNSW-NB15} that mentioned near-perfect detection scores which makes us wonder whether they have any limitations. Most of the works \cite{nawir2019effective} \cite{Kanimozhi2019UNSW-NB15}, \cite{meghdouri2018analysis}  followed only one validation setup, making it impossible to compare their works with the ones, having the same dataset with different validation setup.



Our work differs from the previous ones in the the following ways.
\begin{itemize}
    \item We have provided a through study of the UNSW-NB15 dataset with feature engineering, preprocessing, selection. In earlier studies, we did not find any mention of feature engineering to improve results.
    \item We have explored the performance of a boosting algorithm in binary classification on the dataset following different experimentation setups from prior studies. In prior studies, we have found each work focused on only one evaluation process.
    \item We have compared our results with prior state-of-the art techniques  with respect to all related performance metrics.
\end{itemize}

% Brief summary of major findings of results
Our results show that  feature engineering can make the model more generalized, thereby improving performance on separate test data.     Our proposed model can better predict network anomaly than normal records. Our accuracy and f1\_score are 91.95\% and 92.96\%, respectively on the separate test dataset.


%Application of this work
Our work can help in detecting  unseen anomaly attacks better having very few false alarms.  Our different experimentation setups will help visualize the impact of validation strategies on the model performance of this dataset.


The rest of the paper is organized as follows. In Section~\ref{relatedWorks},  recent  works related to NIDS on the UNSW-NB15 dataset are listed. Our proposed methodology is explained in
Section ~\ref{methodology}. In Section \ref{results}, we have described the experimentation setups and our results as well as some comparisons with the prior state-of-the-arts.
The rest of the comparisons regarding evaluating on train and test data, cross validation approaches are shown in Section
\ref{comparison}. Finally, Section~\ref{conclusion} has the concluding remarks.



%---- Related works ------------------------
%---- Related works ------------------------
%---- Related works ------------------------

\section{Related works}
\label{relatedWorks}

For network intrusion detection KDDCUP99, NSL-KDD, DARPA, UNSW-NB15 are among the benchmark dataset. As a popular dataset, we focus on binary classification of UNSW-NB15 dataset~\cite{moustafa2015unsw}  which is used in  several anomaly detection works. Based on model evaluation process, we have divided them into several parts.

\subsection{Random train test}
Moustafa et al.\cite{moustafa2017hybrid} used central points of attribute values and Association Rule Mining for feature
selection on a high level of abstraction from datasets UNSW-NB15 and NSL-KDD. They partitioned the datasets into train test following an equation and evaluated performance using Expectation-Maximisation clustering (EM), Logistic Regression (LR) and Naïve Bayes (NB).
%The LR produced the best results on the two datasets with accuracy and FAR, 83\% and 14.2\% on UNSW-NB15 dataset, 82.1\% and 17.5\% on NSL-KDD dataset.
Moustafa et al. \cite{moustafa2018anomaly} also proposed a beta mixture
model-based anomaly detection system on the UNSW-NB15 dataset. They first selected eight features from the dataset,
then randomly selected samples from it.
% The best result had accuracy 93.4\%, detection rate 92.7\% and false positive rate 5.9\%.
In another work, Moustafa et al. \cite{moustafa2019holistic} selected random samples from
the UNSW-NB15 dataset and ran ten-fold cross-validation on it. % They found the LogisticRegression classifier to achieve the best result with 95.6\% accuracy and 5.6\% false alarm rate.

\subsection{Validation on same data used for training}
Mogal et al.\cite{mogal2017nids} used machine learning classifiers on both UNSW-NB15 and KDDCUP99 datasets.
They achieved nearly 100\% accuracy on both datasets using Naive Bayes and Logistic Regression on train data. Kanimozi et al. \cite{Kanimozhi2019UNSW-NB15} chose the best four features of
the UNSW-NB15 dataset using the RandomForest classifier. They also used a Multi Layer Perceptron to show how neural
networks would perform on this dataset.

\subsection{Cross validation}
Koroniotis et al.\cite{koroniotis2017towards} selected the top ten features of the UNSW-NB15 combined (train+test)
dataset using Information Gain Ranking Filter. Then they ran ten-fold cross-validations using machine learning techniques.
Among the four techniques (ARM, DT, NB, ANN) applied, DT (Decision Tree C4.5 Classifier) performed the best at
 distinguishing between Botnet and normal network traffic.

 Suleiman et al. \cite{suleiman2018performance} explored the performance of machine learning classifiers on benchmark
and new dataset (UNSW-NB15, NSL-KDD, and Phishing) using ten-fold cross-validation. They found the RandomForest
classifier to perform best. All the experiments were done using the WEKA tool.
% on the NSL-KDD dataset with accuracy 99.76\%, on UNSW-NB15 with accuracy 90.14\%. For the Phishing dataset J48 classifier performed best with accuracy 90.76\%.

Nawir et al. \cite{nawir2019effective} applied ten-fold cross-validation on the binary classification of the combined
(train+test) dataset by using the WEKA tool. They also compared centralized and distributed AODE algorithms based on
accuracy against the number of nodes.

 Meftah et al. \cite{meftah2019network} applied both binary and multiclass
classification on the UNSW-NB15 dataset. They found for binary classification SVM performs the best in ten-fold
cross-validation and decision tree (C5.0) for multiclass classification.
% with 82.11\% accuracy and decision tree (C5.0) perform the best for multiclass classification with 86\% f-measure and 85.41\% accuracy on train data.
Hanif et al. \cite{hanif2019intrusion} used ANN(Artificial Neural Network)
on the same dataset. The neural network had one hidden layer and it achieved an average 84\% accuracy and less
than 8\% false-positive rate in repeated cross-validation. They compared their performance with prior works on the
NSL-KDD dataset, instead of works on the same dataset.

 Meghdouri et al. \cite{meghdouri2018analysis} applied feature preprocessing and principal component analysis on the
 UNSW-NB15 dataset. Then performed five-fold cross-validation using a RandomForest classifier.
 % and achieved 84.9\% f-measure.

\subsection{Validation on separate test data}
In \cite{moustafa2016evaluation} Moustafa et al. analyzed the statistical properties of the UNSW-NB15 dataset. The complexity of the dataset was evaluated using
five techniques (DT, LR, NB, ANN and EM clustering). Here, based on the performance results UNSW-NB15 was found to be
more complex compared to the KDD99 dataset.

Vinaykumar et al. \cite{vinayakumar2019deep} used both classical machine learning classifiers and deep neural networks on several intrusion detection datasets. The classical models performed much better than the neural network models.
% Their best DNN performance on test dataset was 79.4\% accuracy and 80.2\% f1\_score. However, their RandomForest model achieved 90.2\% accuracy and 92.4\% f1\_score which is far better than the deep neural network model.
Dahiya et al. \cite{dahiya2018network} applied feature reduction techniques (CCA, LDA) on train and test dataset. They worked on both larger and smaller version of UNSW-NB15 dataset. Bhamare et al. \cite{bhamare2016feasibility} tested the robustness of machine learning models in cloud scenarios. So they trained classifiers on UNSW-NB15 dataset and tested them on a cloud security dataset ISOT. They found that these models do not perform well in a different environment.

\subsection{Others}
Viet et al. \cite{viet2018using} used deep learning model on the UNSW-NB15 and NSL-KDD datasets only to detect network scanning attacks. In UNSW-NB15 as the scanning types are
labelled altogether, so they applied binary classification for it.
% Using Deep Belief Network, they achieved TPR and FAR, 99.86\% and 2.76\%  on UNSW-NB15 dataset, 99.458\% and 2.71\% on NSL-KDD dataset.
% Moustafa et al. \cite{moustafa2015significant} studied the significant features of the UNSW-NB15 and KDD99 dataset. The results show the original KDD99 attributes are less efficient than the replicated UNSW-NB15 attributes of the KDD99 data set.
% Belouch et al. \cite{belouch2018performance} used machine learning classifiers to detect network intrusions on the UNSW-NB15 dataset on Apache Spark.
% Their RandomForest classifier achieved 97.49\% accuracy on the train dataset.

To the best of our knowledge, there has been no work that  provided a through study of the UNSW-NB15 dataset with feature engineering to improve results. Moreover, most of the previous works focused on only one evaluation process.
We have used feature engineering to reduce overfitting, thereby  providing a generalized model. Moreover, we have evaluated our model performance using multiple experimentation setups and provided thorough comparison with prior state-of-the-art techniques.



%--------------------------------------------

\section{Proposed Methodology} \label{methodology}
We have targeted only to perform binary classification on the dataset. Kaggle kernels were used for running our models.
It provided with 4 CPU cores, 16 Gigabytes of RAM when this work was done. In following subsections,  we described how the dataset was prepared for experimentation and the performance metrics used for evaluation.


\subsection{Dataset Description}
We have used the UNSW-NB15 dataset\cite{moustafa2015unsw} which is a recent benchmark dataset for NIDS.
It was designed at the Cyber Range Lab of the Australian Center of Cyber Security. Compared to other existing datasets (such as KDDCup99, NSL-KDD, DARPA),  UNSW-NB15 dataset is more recent and better reflects modern network traffic. UNSW-NB15 represents nine major families of attacks by utilizing the IXIA PerfectStorm tool.
The main data set contains 2,540,044 observations. A part of this data set was
divided into train and test sets by the authors, which has been used in this work. The dataset description is shown in Table \ref{datasetDescription}. We have considered binary classification for this study.
Hence, we have only predicted whether the record is attack type or normal.

\begin{table}
\normalsize
\centering
\caption{Dataset Description}
\label{datasetDescription}
\renewcommand{\arraystretch}{1.2}

\begin{tabular}{|C{3cm}|C{2cm}|C{2cm}|}
\hline
 \textbf{Type} & \textbf{Train} & \textbf{Test} \\ \hline
Normal & 56,000 & 37,000 \\ \hline
Anomaly & 119,341 & 45,332 \\ \hline
\textbf{Total}  & \textbf{175,341} & \textbf{82,332} \\ \hline
\end{tabular}
\end{table}


\subsection{Preprocessing \label{preprocessing}}
We have performed the preprocessing on the data set using the following steps :

\subsubsection{Dropping unnecessary columns}
We have dropped id column data as it can not be learned as a feature. Also, attack\_cat as we are not doing multi-class
classification. And label column is our target. So among 45 features, 42 were left after dropping these three.

\subsubsection{Feature engineer categorical columns}
We have found many categorical labels to have very low frequency. To make it easier for the model to learn from these
categorical features, labels with low frequency were converted into a single label. For state feature, except
the top five labels by frequency ('FIN', 'INT', 'CON', 'REQ', 'RST') other labels were converted into label 'others'.
Similarly, for service column labels except '-', 'dns', 'http', 'smtp', 'ftp-data', 'ftp', 'ssh', 'pop3' were converted
into 'others' label. For proto column 'igmp', 'icmp', 'rtp' labels were combined into label 'igmp\_icmp\_rtp'.
Then labels except 'tcp', 'udp', 'arp', 'ospf', 'igmp\_icmp\_rtp' were converted into label 'others'. Before this, test data had new categorical values present. However, after this preprocessing, categorical value sets for train and test became same.


\subsubsection{Scaling}
We have applied StandardScaler from sklearn's preprocessing library on all non-categorical features. It was
fitted on the train data, then the fitted scaler was used to covert both train and test data. It converts values
using the following equation:
\begin{equation}
    x = \frac{x-\mu}{\sigma}
\end{equation}

where $\mu$ is the mean value and $\sigma$ is the standard deviation.

\subsubsection{Feature Selection}
We have used the RandomForest classifier of sklearn with default parameters to calculate feature importance on the
train dataset. We have first preprocessed dataset using previous steps. Then averaged feature importance over ten-fold
cross-validation. We have converted the values into percentages, for easier understanding. Then sorted them in descending order.
From there we have chosen to drop features with less than 0.5\% importance value. The dropped 7 features are response\_body\_len,
spkts, ct\_flw\_http\_mthd, trans\_depth, dwin, ct\_ftp\_cmd, is\_ftp\_login. In table \ref{featureImportance}
we have shown the chosen features with corresponding importance.

\begin{table}
\normalsize
\centering
\caption{Feature Importance}
\label{featureImportance}
% increases cell padding
\renewcommand{\arraystretch}{1.2}
\begin{tabular}{|C{2.1cm}|C{1.1cm}|C{2.5cm}|C{1.1cm}|}
\hline
\textbf{Feature} & \textbf{Importance} & \textbf{Feature} & \textbf{Importance}\\ \hline

sttl & 16.53& sjit & 1.7  \\ \hline
ct\_state\_ttl & 11.06& dloss & 1.22  \\ \hline
dload & 7.2& proto & 1.21  \\ \hline
dttl & 4.93& djit & 1.14  \\ \hline
dmean & 4.19& sloss & 0.9  \\ \hline
ackdat & 3.8& ct\_src\_ltm & 0.83  \\ \hline
rate & 3.79& ct\_dst\_ltm & 0.83  \\ \hline
dinpkt & 3.51& stcpb & 0.81  \\ \hline
sbytes & 3.23& ct\_dst\_sport\_ltm & 0.8  \\ \hline
smean & 2.85& dtcpb & 0.75  \\ \hline
sload & 2.72& swin & 0.62  \\ \hline
state & 2.71& is\_sm\_ips\_ports & 0.57  \\ \hline
dpkts & 2.59& ct\_src\_dport\_ltm & 0.57  \\ \hline
tcprtt & 2.49& service & 0.51  \\ \hline
ct\_srv\_dst & 2.49& spkts & 0.47  \\ \hline
ct\_dst\_src\_ltm & 2.43& ct\_flw\_http\_mthd & 0.17  \\ \hline
sinpkt & 2.41& response\_body\_len & 0.16  \\ \hline
ct\_srv\_src & 2.2& trans\_depth & 0.14  \\ \hline
dbytes & 1.93& dwin & 0.02  \\ \hline
synack & 1.76& ct\_ftp\_cmd & 0.01  \\ \hline
dur & 1.76& is\_ftp\_login & 0.01  \\ \hline

\end{tabular}
\end{table}


\subsubsection{OneHotEncoding}
We have used pandas library to OneHotEncode all the categorical features. It became possible as after using feature engineering, categorical value sets became same in train and test dataset. The final number of features in our dataset is 53.


\subsection{Evaluation metrics}
Here, we have discussed all the performance metrics used to compare the performance of our approach with previous works.
%All the metrics were calculated using the sklearn.metrics module.

\begin{itemize}
    \item \textbf{True Positives (TP)}: The cases in which YES was predicted and the actual output was also YES.
    \item \textbf{True Negatives (TN}: The cases in which NO was predicted and the actual output was NO.
    \item \textbf{False Positives (FP)}: The cases in which YES was predicted and the actual output was NO.
    \item \textbf{False Negatives (FN)}: The cases in which NO was predicted and the actual output was YES.
\end{itemize}

\subsubsection{Accuracy}
It is the ratio of the number of correct predictions to the total number of input samples.
\begin{equation}
    Accuracy = \frac{TP+TN}{TP+TN+FP+FN}
\end{equation}

\subsubsection{Precision}
It is ratio of the number of correct positive results to the number of positive results predicted by the classifier.
\begin{equation}
    Precision = \frac{TP}{TP+FP}
\end{equation}

\subsubsection{Recall or detection rate or True Positive Rate)}
It is the ratio of the number of correct positive results to the number of all relevant samples (all samples that should
have been identified as positive).
\begin{equation}
    recall = \frac{TP}{TP+FN}
\end{equation}

\subsubsection{F1\_score}
The harmonic mean of precision and recall.
\begin{equation}
    f1\_score = 2 * \frac{1}{\frac{1}{precision}+ \frac{1}{recall}}
\end{equation}

\subsubsection{False Positive Rate (FPR)}
It is the proportion of incorrectly identified observations.
\begin{equation}
    FPR = \frac{FP}{FP+TN}
\end{equation}

\subsubsection{False Alarm Rate (FAR)}
It represents the probability that a record gets incorrectly classified.
\begin{equation}
    FAR = \frac{FP+FN}{FP+FN+TP+TN}
\end{equation}

\subsubsection{ROC AUC}
It computes the Area Under the Receiver Operating Characteristic Curve (ROC AUC) from prediction scores.

\subsubsection{Time (sec)}
As run time is an important factor in detecting network anomalies, we have determined the time required to run our models.
%We have used the time library of python for this purpose.


\section{Experiment and Results \label{results}}
For evaluating the UNSW-NB15 dataset, ten-fold cross validation was performed on the train dataset using Stratified KFold of the sklearn library with random shuffle true. We have used several popular machine learning classifiers to measure the
detection performance. The models were run mostly with default parameters. We have set random state to 1 for all of them,
so that the results are reproducible. All models except LightGBM \cite{ke2017lightgbm}, were from sklearn library
version 0.23.0. During prediction, for LightGBM we used the best iteration. The used models are listed below with
important parameters.
\begin{enumerate}
    \item LogisticRegression : penalty = l2, max\_iter = 100, solver = lbfgs, C = 1.0
    \item GradientBoosting: learning\_rate = 0.1, n\_estimators = 100,max\_depth = 3
    \item DecisionTree: criterion = 'gini', max\_depth = None, max\_features = None,
    \item RandomForest: n\_estimators = 100, criterion = 'gini', max\_depth = None, max\_features = None
    \item LightGBM: learning rate = 0.1, objective = binary, metric = binary\_logloss,boost\_from\_average = True,
    num\_round = 2000, early\_stopping\_rounds = 50.
\end{enumerate}

Then, we have chosen the best model based on accuracy and f1-score. As shown in Table \ref{crossvalidationWithDifferentModels},
LightGBM showed the best performance in both accuracy (96.18\%) and f1-score (91.21\%).
This is because LightGBM follows a more complex leaf-wise split approach rather than a level-wise approach. Hence, it reduces overfitting on train data and improves the validation results.

\begin{table}
\normalsize
\centering
\caption{Ten-fold cross validation with different models}
\label{crossvalidationWithDifferentModels}
\renewcommand{\arraystretch}{1.2}

\begin{tabular}{|C{2.8cm}|C{2.1cm}|C{2.1cm}|}
\hline
\textbf{Metrics(\%)} & \textbf{Accuracy(\%)} & \textbf{F1\_score(\%)} \\ \hline
LogisticRegression & 93.54 & 95.42 \\ \hline
GradientBoosting & 94.58 & 96.11\\ \hline
DecisionTree  & 94.99 & 96.32\\ \hline
RandomForest  & 96.08 & 97.14 \\ \hline
LighGBM & 96.18 & 97.21 \\ \hline
\end{tabular}
\end{table}


\subsection{Validation on same data used for training}
Mogal et al. \cite{mogal2017nids}, Kanimozhi et al. \cite{Kanimozhi2019UNSW-NB15} evaluated model performance on the
UNBSW-NB15 dataset without using any cross-validation approach. The same data used for training the model was used for
validation too. To compare our model's performance with them, we have followed the similar setup. As evident from the
results shown in table \ref{evaluationOnTrainData}, this type of experimentation setup does not truly reflect model
performance. As the model overfits on train data, its performance will become very poor on a separate test set. For
example, we have found our model when overfitted on train data, only achieved 86.88\% accuracy and 89.14\% f-measure on
test data.

\begin{table}
\normalsize
\centering
\caption{Evaluating model on data used for training}
\label{evaluationOnTrainData}
\renewcommand{\arraystretch}{1.2}

\begin{tabular}{|C{3.5cm}|C{1.8cm}|C{1.8cm}|}
\hline
\textbf{Metrics(\%)} & \textbf{Train} & \textbf{Test} \\ \hline
Accuracy & 99.60 & 99.98 \\ \hline
Precision & 99.52 & 99.97\\ \hline
Recall  & 99.89 & 99.98\\ \hline
F1\_score  & 99.71 & 99.98 \\ \hline
FPR & 0.01 & 0.0004\\ \hline
AUC & 99.99 & 99.99\\ \hline
Time(s) & 243 & 237\\ \hline
\end{tabular}
\end{table}


\subsection{Ten-fold cross validation}
Ten-fold cross-validation on train, test or combined(train+test) dataset was performed in \cite{meftah2019network},
\cite{suleiman2018performance}, \cite{nawir2019effective}, \cite{hanif2019intrusion} . We have used the StratifiedKFold
method of sklearn.model\_selection module with shuffle = True to perform the ten-fold cross validation. Average
scores achieved in that process is shown in table \ref{tenFoldCrossValidation}.

\begin{table}
% this increases the fontsize used in table
\normalsize
\centering
\caption{Ten-fold cross validation}
\label{tenFoldCrossValidation}
% increases cell padding
\renewcommand{\arraystretch}{1.2}
\begin{tabular}{|C{2.3cm}|C{1.3cm}|C{1.3cm}|C{1.8cm}|}
\hline
\textbf{Metrics(\%)} & \textbf{Train} & \textbf{Test} & \textbf{Combined}\\ \hline
Accuracy & 96.18 & 98.18 & 95.19 \\ \hline
Precision & 96.54 & 98.87& 96.84\\ \hline
Recall  & 97.89 & 97.80 & 95.58\\ \hline
F1\_score  & 97.21 & 98.33 & 96.21\\ \hline
FPR & 7.47 & 1.37 & 5.51\\ \hline
FAR & 3.82 & 1.83 & 4.81 \\ \hline
AUC & 99.45 & 99.82 & 99.26\\ \hline
Time(s) & 628.1 & 281.1 & 838.8\\ \hline
\end{tabular}
\end{table}


\subsection{Validation on test data \label{validationResultsOnTest}}
In this phase we have validated the model trained on train data using the separate test dataset of UNSW-NB15 following
\cite{moustafa2016evaluation} \cite{bhamare2016feasibility} \cite{vinayakumar2019deep} \cite{dahiya2018network}. As  Meftah et al. \cite{meftah2019network}
mentioned, some columns have new labels in test data. However, after our feature engineering process in section
\ref{preprocessing}, we were able to overcome it. For this evaluation specifically, we have found setting parameter
is\_unbalance: True and learning rate to 0.05 in LightGBM improved prediction performance. The performance metrics
are shown in table \ref{validationResult} along with comparisons with prior arts. In table \ref{confusionMatrix}
we have shown the normalized confusion matrix.

\begin{table}
\normalsize
\centering
\caption{Validation on test data}
\label{validationResult}
\renewcommand{\arraystretch}{1.2}
\begin{tabular}{|C{2cm}|C{1.2cm}|C{1.3cm}|C{1.7cm}|}
\hline
\textbf{Metrics(\%)} & \textbf{Ours} & \textbf{RF\cite{vinayakumar2019deep}} & \textbf{REP Tree\cite{dahiya2018network} }\\ \hline
Accuracy & 91.95 & 90.3 & 93.56 \\ \hline
Precision & 89.59 & 98.8 & 83.3\\ \hline
Recall  & 96.60 & 86.7 & 83.2 \\ \hline
F1\_score  & 92.96 & 92.4 & 83.25 \\ \hline
FPR & 13.75 & - & 2.3 \\ \hline
FAR & 8.05 & - &  - \\ \hline
AUC & 98.64 & - & - \\ \hline
Time(s) & 31.44 & - & - \\ \hline

\end{tabular}
\end{table}

Our model outperforms the work of Vinayakumar et al \cite{vinayakumar2019deep} by both accuracy and f1\_score.
Though Dahiya et al \cite{dahiya2018network} achieved better accuracy than ours, they had near 10\% drop in
1\_score than our model. In intrusion detection dataset where class distribution is imbalanced, f1\_score is more important.

\begin{table}
\normalsize
\centering
\caption{Confusion matrix}
\label{confusionMatrix}
\renewcommand{\arraystretch}{1.2}
\begin{tabular}{|C{2.4cm}|C{2.2cm}|C{2.2cm}|}
\hline
 & Predicted Normal & Predicted Anomaly \\ \hline
Actual Normal & 0.86 & 0.14 \\ \hline
Actual Anomaly & 0.03 & 0.97\\ \hline
\end{tabular}
\end{table}


\section{Comparison with state-of-the-art models} \label{comparison}
In this section, we compare our model performance with prior state-of-the-art models on same dataset. We have arranged this section in subsections based on different experimentation setups that was followed in previous works.


\subsection{Evaluation on train data}
Mogal et al.~
\cite{mogal2017nids} achieved 99.96\% accuracy on the UNSW-NB15 dataset using Naive Bayes and Logistic Regression,
which did not follow any cross-validation approach. A similar approach was taken by Kanimozhi et al.~\cite{Kanimozhi2019UNSW-NB15} with the best four features chosen using the RandomForest classifier.
The model achieved 98.3\% accuracy. It is shown in Table~\ref{evaluationOnTrainData} that in the same validation process, our model achieves near-perfect scores on the train and test data. We also did not find any comparison to
prior state-of-the-art with similar validation process\cite{mogal2017nids} in \cite{Kanimozhi2019UNSW-NB15}.

\subsection{Ten-fold cross validation}
Suleiman et al.~\cite{suleiman2018performance} evaluated performance using ten-fold cross validation on train data.
They found a Random Forest (RF) classifier to have the best accuracy and f-measure. We have mentioned our model performance
using same validation process in column train of Table~\ref{tenFoldCrossValidation}. TPR and recall are same; hence, we have mentioned only recall.


\begin{table}
\normalsize
\centering
\caption{Performance comparison with \cite{suleiman2018performance}}
\label{performanceComparisonWithSuleiman}
\renewcommand{\arraystretch}{1.2}
\begin{tabular}{|C{2.3cm}|C{2.2cm}|C{2.2cm}|}
\hline
\textbf{Metrics(\%)} & \textbf{RF \cite{suleiman2018performance}} & \textbf{LightGBM} \\ \hline
Accuracy & 90.14 & 96.17\\ \hline
Precision  & 99.8 & 96.54\\ \hline
Recall  & 97.8 & 97.89\\ \hline
F1\_score  & 98.7 & 97.20\\ \hline
FPR  & 0.1 & 7.48\\ \hline
\end{tabular}
\end{table}

Meftah et al. \cite{meftah2019network} applied ten-fold cross validation on train dataset and achieved the best accuracy
82.11\% using SVM classifier. In same validation process, our model accuracy is 96.17\%. Hanif et al.~\cite{hanif2019intrusion}
applied ten-fold cross validation on the train and test dataset repeatedly using Artificial Neural Network(ANN) and achieved
average 84\% accuracy, 8\% false positive rate. In similar case, our model performance is better (having 96.18\% accuracy and 7.47\% FPR as shown in Table~\ref{tenFoldCrossValidation}). Though \cite{meftah2019network}\cite{hanif2019intrusion} followed the same experimentation
setup similar to \cite{suleiman2018performance}, they did not present any comparisons.

Koroniotis et al.~\cite{koroniotis2017towards} performed ten-fold cross validation on the combined dataset.
The best result was achieved using the Decision Tree C4.5. Our model has shown better performance, having higher accruacy and lower false alarm rates ( see Table \ref{performanceComparisonWithKoroniotis}).

\begin{table}
\normalsize
\centering
\caption{Comparison of our model with Koroniotis et al. \cite{koroniotis2017towards}}
\label{performanceComparisonWithKoroniotis}
\renewcommand{\arraystretch}{1.2}

\begin{tabular}{|C{3.3cm}|C{2.2cm}|C{1.5cm}|}
\hline
\textbf{Classifier} & \textbf{Accuracy (\%)} & \textbf{FAR(\%)} \\ \hline
Decision Tree \cite{koroniotis2017towards} & 93.23 & 6.77 \\ \hline
LightGBM & 95.19 & 4.81 \\ \hline
\end{tabular}
\end{table}

Nawir et al. \cite{nawir2019effective} applied a similar ten-fold cross-validation evaluation on the combined dataset
using the WEKA J48 classifier. They mentioned achieving high accuracy of 98.71\% using the default parameter.
However, using exactly the same environment for several runs we have found it achieves around 94.6\% accuracy,
which is lower than ours (95.19\%  accuracy).

\subsection{Five-fold cross validation}
We have found only Meghdouri et al. \cite{meghdouri2018analysis} to validate using five-fold cross-validation.
So we did not add any separate section for it, however we have presented our model performance using same validation process
here in table \ref{performanceComparisonWithMeghdouriTrain} and \ref{performanceComparisonWithMeghdouriTest}.
Table \ref{performanceComparisonWithMeghdouriTrain} shows our model performance compared to their's on five-fold
cross validation of train dataset.

Their model achieved higher accuracy (99\%) compared to ours (96.18\%). However, for precision, recall and
f1\_score our model performance is much higher. For same validation process on test dataset from
table \ref{performanceComparisonWithMeghdouriTest}, our test accuracy is very close to theirs.
However, as before our precision, recall and f1\_score achieved much better than theirs. Our ROC AUC
scores are very close too. However, for intrusion detection techniques f-measure is very important,
in which our model outperforms by a large margin.

\begin{table}
\normalsize
\centering
\caption{Comparison with Meghdouri et al.\cite{meghdouri2018analysis} (Train data)}
\label{performanceComparisonWithMeghdouriTrain}
\renewcommand{\arraystretch}{1.2}
\begin{tabular}{|C{3cm}|C{2cm}|C{2cm}|}
\hline
\textbf{Metrics(\%)} & \textbf{Train\cite{meghdouri2018analysis}} & \textbf{Train} \\ \hline
Accuracy & 99.0 & 96.18\\ \hline
Precision  & 85.9 & 96.56 \\ \hline
Recall  & 85.1 & 97.87 \\ \hline
F1\_score  & 84.9 & 97.21 \\ \hline
ROC AUC  & 99.8 & 99.44 \\ \hline
\end{tabular}
\end{table}

\begin{table}
\normalsize
\centering
\caption{Comparison with Meghdouri et al.\cite{meghdouri2018analysis} (Test data)}
\label{performanceComparisonWithMeghdouriTest}
\renewcommand{\arraystretch}{1.2}
\begin{tabular}{|C{3cm}|C{2cm}|C{2cm}|}
\hline
\textbf{Metrics(\%)} & \textbf{Test\cite{meghdouri2018analysis}} & \textbf{Test} \\ \hline
Accuracy & 98.9 & 98.08\\ \hline
Precision  & 84.9 &  98.79\\ \hline
Recall & 85.1 & 97.7\\ \hline
F1\_score & 84.9 & 98.24 \\ \hline
ROC AUC   & 99.8& 99.81\\ \hline
\end{tabular}
\end{table}

\subsection{Validation on separate test data}
Bhamare et al. \cite{bhamare2016feasibility} achieved acc 89.26\%, TP 93.7\% and TN 95.7\% at prediction threshold 0.5.
Increasing prediction threshold to 0.7-0.8 their TP improved to 97\%, but TN dropped to 80\%. Where our TP and
TN are 97\% and 86\% at threshold 0.5 as shown in table \ref{confusionMatrix}.
Moustafa et al. \cite{moustafa2016evaluation} achieved 85.56\% accuracy and 15.78\% FAR using
DT technique built in Visual Studio Business Intelligence 2008 with the default input parameters. Our model achieved 91.95\% accuracy and 8.05\% FAR in this case.

\subsection{Others}
We have not compared with some prior arts. Despite having a separate train and test dataset where authors chose to
randomly sample train and test data. For example Moustafa et al. \cite{moustafa2017hybrid} \cite{moustafa2018anomaly} \cite{moustafa2019holistic}
evaluated the model on randomly chosen data from UNSW-NB15 dataset. However, it is not possible to reproduce a random dataset. Also we have not compared our model run time with prior arts as the run time environments are not same.

% Belouch et al. \cite{belouch2018performance} found RandomForest classifier to achieve 97.49\% accuracy, 93.53\% sensitivity and 97.75\% specificity. However, it was not clear which validation approach was taken.


\subsection{Results summary}

Followings are the summary of our results:
\begin{itemize}
    \item Validating on same data used for training the model would give near perfect results. However, it is due to overfitting which should be avoided.
    \item Feature engineering can make the model more generalized, thereby improving performance on separate test data.
    \item There are many features having very low importance. It is found that 17 features have importance less than 1\%.
    \item Our model can better predict network anomaly than normal records. This can be due to the presence of more anomaly in dataset than normal.
\end{itemize}

\section{Conclusion \label{conclusion}}
In this paper, we have presented a boosting algorithm-based model for performing binary classification of the UNSW-NB15 dataset.
%We have explained all the steps taken from feature preprocessing, selection and validation of the model.
Different experimentation setups were followed to compare our performance with prior works. Results show that our trained model outperforms state-of-the-art works in most metrics. We have  shown why the experimental setups followed by
some prior state-of-the-art works are heavily overfitted and should be avoided. Our model is found to perform well on test data when
it is fitted on train data only, validating the generalization of our model.

This study only performs a binary classification. In  future, we intend to improve  the performance of multiclass-classification on this dataset in a similar way.


\bibliography{bibliography}
\end{document}


